\documentclass[11pt,a4paper,twoside,pdf]{article}

% Paquetes (añade otros si los necesitas):

\usepackage[T1]{fontenc}
\usepackage[utf8]{inputenc}


\usepackage[pdftex,final]{graphicx}
\bibliographystyle{plain} % We choose the "plain" reference style

% Style package
% Font Package (Palatino)
\usepackage{mathpazo}
% Packages for specific capabilities
\usepackage{rotating} % for text rotation in tables
\usepackage{multirow} % for multirow in tables
\usepackage{subfigure} % Place subfigures in figure environment
% Packages for specific symbols
\usepackage{amssymb}
\usepackage{amsmath}
\usepackage{amsfonts}
\usepackage{braket}
\usepackage{eurosym} % Euro symbol
\usepackage{bbding} % for \XSolidBrush
\usepackage{pifont} % for \ding{55} (a check mark)
\usepackage{macro}


\usepackage{latexsym}
\usepackage{comment}
\usepackage{soul}
\usepackage{array}
%\usepackage{marvosym}
\usepackage{epsfig}
\usepackage{graphics}
\usepackage{amsfonts}
\usepackage{xspace}
\usepackage{color}
\usepackage{booktabs}
\usepackage{xtab}
\usepackage[colorlinks=true,urlcolor=blue,linkcolor=blue,citecolor=blue]{hyperref}
\numberwithin{equation}{section}


\linespread{1.05}

% TFG en inglés:
\usepackage[english]{babel} 
\addto\captionsenglish{\renewcommand{\chaptername}{}}

% TFG en español:
%\usepackage[spanish,es-nodecimaldot,es-tabla,es-lcroman,es-nosectiondot,es-noindentfirst]{babel}
%\renewcommand\spanishchaptername{}

% Formato de la página:
\usepackage{fancyhdr}
\usepackage[top=2.88cm,bottom=2.97cm,left=2.95cm,right=2.95cm]{geometry}
\setlength{\parskip}{0.1cm}

% Pon aquí tus definiciones:

\newcommand{\dis}{\displaystyle}
%\sodef\an{}{.2em}{1em plus1em}{2em plus.1em minus.1em}

\begin{document}

% Portada %%%%%%%%%%%%%%%%%%%%%%%%%%%%%%%%%%%%%%%%%%%%%%%%%%%%%%%%%%%%%%%%%%%%%%

\pagestyle{empty}


\noindent
\begin{tabular}{r}
\includegraphics[width=8.8cm]{escudoUGRmonocromo.png} \\[-1.8ex]
\hspace{31mm}\vspace{-8mm}
\begin{tabular}{c}
\hline\\[-1ex]\hskip-2mm
{\bf Facultad de Ciencias}\hspace{18mm}
\end{tabular}
\end{tabular}

\large
\vspace{30mm}
\hspace{25mm}
\begin{tabular}{l}
GRADO EN F\'ISICA
\end{tabular}

\vspace{45mm}
\hspace{25mm}
\begin{tabular}{l}
TRABAJO FIN DE GRADO
\\[1.5ex]
\LARGE\bf Herramientas para cálculos \\ 
\LARGE\bf perturbativos en renormalización \\
\LARGE\bf de Hamiltonianos \\
\end{tabular}
%
\vfill
\hspace{25mm}
\begin{tabular}{l}
Presentado por:
\\
{\bf D. ZhuoZhuo Liu}
\\[3ex]
Curso Académico 2024/2025
\end{tabular}
%

\newpage
%
\begin{center}
{\bf Resumen}
\bigskip

\begin{minipage}{0.8\linewidth}
bla bla bla bla bla bla bla bla bla bla bla bla bla bla bla
bla bla bla bla bla bla bla bla bla bla bla bla bla bla bla
bla bla bla bla bla bla bla bla bla bla bla bla bla bla bla 
bla bla bla bla bla bla bla bla bla bla bla bla bla bla bla
bla bla bla bla bla bla bla bla bla bla bla bla bla bla bla
\end{minipage}

\newpage

{\bf Abstract} 
\bigskip

\begin{minipage}{0.8\linewidth}
bla bla bla bla bla bla bla bla bla bla bla bla bla bla bla
bla bla bla bla bla bla bla bla bla bla bla bla bla bla bla
bla bla bla bla bla bla bla bla bla bla bla bla bla bla bla 
bla bla bla bla bla bla bla bla bla bla bla bla bla bla bla
bla bla bla bla bla bla bla bla bla bla bla bla bla bla bla
\end{minipage}

\newpage

\end{center}

% Indice %%%%%%%%%%%%%%%%%%%%%%%%%%%%%%%%%%%%%%%%%%%%%%%%%%%%%%%%%%%%%%%%%%%%%%%
%\newpage

\tableofcontents

% Texto %%%%%%%%%%%%%%%%%%%%%%%%%%%%%%%%%%%%%%%%%%%%%%%%%%%%%%%%%%%%%%%%%%%%%%%%
%\newpage

\pagestyle{fancy}
\fancyhead[RO,LE]{\leftmark}
\fancyhead[LO,RE]{\thepage}
\fancyfoot{}

\newpage

\section{Introduction}

When trying to build a theory that describes the particles and their interactions, 
it's fundamental that the theory is compatible with the 2 pillars of modern physics,
the special relativity and quantum mechanics. The special relativity is a theory 
that's able to describe the behavior of particles at high energies, and quantum 
mechanics is a theory that describes the behavior of particles at small scales.
The combination of these two theories is the basis of the quantum field theory,
building a theory that describes the particles as excitations of a field.

Other important aspect of the theory is that it should be able to describe the 
different phenomena that occur at different scales. At low energies, the dominant 
physics involves, bound states, and the interactions between particles are weak. 
While at high energies, the dominant physics involves scattering processes, and the
interactions between particles are strong. The theory should be able to describe
the transition between these two regimes.

In the context of theoretical physics, the renormalization group procedure is a 
powerful tool to study the behavior of physical systems at different scales. In
the case of quantum field theories, renormalization allows to study the system
at different energy scales by introducing a scale parameter, and by changing 
this parameter, the "resolution" of the system is changed, allowing to focus from 
the smallest details (the short distance behavior) to the largest ones (the grand 
scale behavior). 

The typical renormalization procedures tend to apply to Lagrangian dynamics, but 
in the framework of RGPEP, the renormalization group procedure is applied to 
Hamiltonian dynamics, the benefit of this approach is the ability obtain directly 
the solutions of the system (the spectrum of the theory, and the eigenstates of the 
Hamiltonian). RGPEP introduces a effective Hamiltonian, that describes the system at a
given scale. This effective Hamiltonian is the solution to a differential equation,
that describes the evolution of the effective Hamiltonian with respect to the
scale parameter.

In general, the exact solution of the Hamiltonian is a non-trivial task, and a 
perturbative method is used to obtain the solution, by identifying each order in the
Hamiltonian expansion with a series of diagrams, the next order in the expansion can
be obtained by some operations on the diagrams of previous orders. In the past, 
these diagrams were obtained by hand, but the number of diagrams increases 
exponentially with each order, making the process tedious and error-prone.
The goal of this thesis is to develop a code that automates the process of
obtaining the diagrams, and to obtain the diagrams for a given order in the expansion.

This thesis is organized as follows. Section \ref{sec:theoretical_background} describes
the theoretical background needed to understand the renormalization group procedure
for effective particles, and the Hamiltonian dynamics. Section \ref{sec:cases}
describes the different cases studied in this thesis, mainly the gluon case, where
the simplicity similar of the scalar case is preserved, but the some of the
peculiarities of the QCD theory are present. The section \ref{sec:code} describes the
connection between the diagrams and the objects are defined in the code, as well
as the steps taken to obtain higher order diagrams, that are presented in the section
\ref{sec:diagrams}. Finally, the section \ref{sec:conclusions} presents the conclusions, 
future work and the improvement that can be done to the code.



\section{Theoretical background} \label{sec:theoretical_background}

To describe the relativistic interactions of effective particles in the framework of 
the renormalization group procedure for effective particles, and making use of 
quantum theory in the front form of Hamiltonian dynamics. It's important to 
understand each of the concepts involved in the process.

\subsection{Front form of Hamiltonian Dynamics}

The front form of dynamics introduced by Dirac (1949) \cite{dirac_front_forms_1949} 
offers a couple advantages to the 

The quantization hypersurface considered is,

\begin{equation}
    x^+ = t+z = x^0 + x^3 = 0,
\end{equation}
then the rest of coordinates will be defined as

\begin{equation}
    x^- = x^0-x^3, \quad x^\perp = (x^1, x^2),
\end{equation}

In this set of coordinates, the fundamental quantities 

The Hamiltonian in FF quantization is obtained from the Lagrangian density using 
the Legendre transformation, similar to the procedure explained in the section
\ref{sec:canonical_hamiltonian}, but with respect to the new coordinates.

\subsection{Canonical Hamiltonian}\label{sec:canonical_hamiltonian}

\begin{comment}
\subsubsection{Classical mechanics}

The canonical Hamiltonian is a function of the canonical coordinates $q_i$,
the canonical momenta $p_i$, and time $t$. The Hamiltonian is a function that
describes the total energy of the system, and is obtained from applying the Legendre 
transformation to the Lagrangian of the system, this is defined as,
\begin{equation}
    H(q,p,t) = \sum_i p_i \dot{q}_i - L(q,\dot{q},t),
\end{equation}
where $H$ is the Hamiltonian, $L$ is the Lagrangian, $q_i$ are the canonical
coordinates, $\dot{q}_i$ are the canonical velocities, and $p_i$ are the
canonical momenta. The canonical momenta are defined as,
\begin{equation}
    p_i = \frac{\partial L}{\partial \dot{q}_i},
\end{equation}

The canonical Hamiltonian governs the time evolution of the system via the 
Hamilton equations of motion, which are given by,
\begin{equation}
    \dot{q}_i = \frac{\partial H}{\partial p_i}, \quad \dot{p}_i = -\frac{\partial H}{\partial q_i}.
\end{equation}

\subsubsection{Field theory}

In the case of field theories, a Hamiltonian density can be defined such that 
integrating over space coordinates the Hamiltonian is obtained. 

\begin{equation}
    H = \int d^3x \mathcal{H},
\end{equation}
for simplicity we will call the Hamiltonian density $\mathcal{H}$, as Hamiltonian.

For a classical field $\phi(x)$, with the Lagrangian density $\mathcal{L}
(\phi,\partial_\mu\phi)$, the Hamiltonian density is defined as,

\begin{equation}
    \mathcal{H} = \pi \dot{\phi} - \mathcal{L},
\end{equation}
where $\pi$ is the canonical momentum, defined as,

\begin{equation}
    \pi = \frac{\partial \mathcal{L}}{\partial \dot{\phi}}.
\end{equation}
\end{comment}

\begin{comment}
    

\subsubsection{Quantum field theory}

In quantum mechanics, or quantum field theory, the Hamiltonian becomes an operator 
acting on a Hilbert space. 

Solving the system means obtaining the spectrum of the theory, this is finding the 
eigenenergies of the system. In general this is a non trivial process. 

In field theory calculations the fundamental quantities to consider are the 
energy-momenta, angular momentum, and boosts, due to its symmetries
under the Poincaré group.

The Poincaré group contains the Lorentz group and the translations, the generators of
the Poincaré group are the four-momentum operator $P^\mu$ and the angular momentum
$M^{\mu\nu}$, which satisfy the following commutation relations,
\begin{align}
    \sbrackets{P^\mu, P^\nu} &= 0, \\
    \sbrackets{M^{\mu\nu}, P^\rho} &= 
    i\left( g^{\nu\rho}P^\mu - g^{\mu\rho}P^\nu \right), \\
    \sbrackets{M^{\mu\nu}, M^{\rho\sigma}} &    = i\left( g^{\nu\rho}M^{\mu\sigma} 
    - g^{\mu\rho}M^{\nu\sigma} + g^{\sigma\mu}M^{\nu\rho} - g^{\sigma\nu}
    M^{\mu\rho}\right).
\end{align}

\end{comment}

\subsubsection{Scalar fields}

\subsubsection{Gluon fields}

The Lagrangian density for the gluon fields is given by,
\begin{equation}
    \mathcal{L} = -\frac{1}{2}\text{tr}F^{\mu\nu}F_{\mu\nu}
\end{equation}
where $F^{\mu\nu}$ is the field strength tensor, defined as,
\begin{equation}
    F^{\mu\nu} = \partial^\mu A^\nu - \partial^\nu A^\mu + ig [A^\mu, A^\nu],
\end{equation}
and $A^\mu = A^{a\mu}t^a$, $t^a$ are the generators of the gauge group, and $g$ is the
coupling constant. Verifying the following relations,

\begin{equation}
    [t^a, t^b] = i f^{abc} t^c, \quad \text{tr}(t^a t^b) = \frac{1}{2} \delta^{ab},
\end{equation}

We will be working in the gauge $A^+=0$, in this gauge the Lagrange equations constrain
the component $A^-$ to become, 

\begin{equation}
    A^- = \frac{1}{\partial^+}2 \partial^\perp A^\perp - \frac{2}{\partial^{+2}} ig 
    \sbrackets{\partial^+ A^\perp, A^\perp},
\end{equation}
this way the only degree of freedom left is the transverse component $A^\perp$.

This way the associated energy-momentum tensor reads,

\begin{equation}
    \mathcal{T}^{\mu\nu} = -F^{a\mu\alpha}\partial^\nu A^a_\alpha + 
    \frac{1}{4}g^{\mu\nu}F^{\alpha\beta} F_{\alpha\beta}.
\end{equation}

The Hamiltonian density in FF quantization is obtained from integrating the component
$\mathcal{T}^{+-}$ of the energy-momentum tensor, over the hyperplane $x^+=0$.




\subsection{Fock space}

In order to describe a variable number of particles in our system, the use of the Fock
space is needed. Intoduced by V.A. Fock in 1932 \cite{1932ZPhy...75..622F}, the Fock 
space is a sum of a set of Hilbert spaces, each one corresponding to a different number of
particles in the system. 

The Fock space is defined as the direct sum of tensor products of the single 
particle Hilbert space $\mathbb{H}$,

\begin{equation}
    \mathbb{F}_\nu = \bigoplus_{n=0}^{\infty} S_\nu \mathbb{H}^{\otimes n} = 
    \mathbb{C} \oplus \mathbb{H} \oplus S_\nu(\mathbb{H} \otimes \mathbb{H}) 
    \oplus S_\nu(\mathbb{H} \otimes \mathbb{H}\otimes \mathbb{H}) \oplus \cdots,
\end{equation}
where $S_\nu$ is the symmetrization operator depending on whether the particles described
are bosonic or fermionic, it symmetrizes or antisymmetric the tensors, and 
$\mathbb{C}$ is the complex scalar, corresponding to the states with no particles. 

This way a general state in the Fock space can be expressed as,

\begin{equation}
    \ket{\Psi}_\nu = \ket{\Psi_0}_\nu \oplus \ket{\Psi_1}_\nu \oplus
    \ket{\Psi_2}_\nu \oplus \cdots = a \ket{0}_\nu + \sum_{i=1}a_i \ket{\psi_i}_\nu +
    \sum_{i,j=1}a_{ij} \ket{\psi_i \psi_j}_\nu + \cdots,
\end{equation}
where $\ket{\Psi_0}_\nu$ is the vacuum state, $\ket{\Psi_1}_\nu$ is the one particle
state, $\ket{\Psi_2}_\nu$ is the two particle state, and so on. The coefficients $a_i$
are the amplitudes of the states, in general complex numbers. 

In the case of QCD, a geneal state in the Fock space involves a superposition of all 
possible multiparticle states, build from quarks, antiquarks, and gluons, with the
correct quantum numbers. This way a general state in the Fock space can be expressed as,
quarkonium

\begin{equation}
    \ket{\Psi} = c_1 \ket{q\bar{q}} + c_2 \ket{q\bar{q}g} + c_3 \ket{qqq} 
    + c_4 \ket{gg} + \cdots,
\end{equation}


\subsection{Counterterms}

The counterterms is an additional term added to the initial or bare Hamiltonian 
$\mathcal{H}_0$ to deal with the divergences due to loops, produced during the 
process. This way, ensure that the effective Hamiltonian $\mathcal{H}_t$ remains 
finite at all values of $t$.



\subsection{RGPEP}
The renormalization group procedure for effective particles (RGPEP), as it's name 
indicate is a renormalization group procedure applied to the Hamiltonian formulation.

By considering a series of unitary transformations, the RGPEP is able to construct a 
series of effective Hamiltonians $\mathcal{H}_s$, and the corresponding effective 
particles, by the use of effective particle operators (namely creation and 
annihilation operators) that differs from the canonical ones by the unitary 
transformation $\mathcal{U}_s$,

\begin{equation}
    a_s = \mathcal{U}_sa_0\mathcal{U}_s^\dagger,
    \label{eq:effective_particle_operator}
\end{equation}
labeled by the parameter $s$, the renormalization group parameter. This parameter 
has dimension of length. Physically $s$ has the interpretation of the characteristic 
size of the effective particles.

Due to dimensional and notational reasons that will be explained latter, it's 
convenient to consider the scale parameter $t = s^4$ instead of $s$. 

The effective Hamiltonian is related to the regulated canonical one with counter-terms 
by the condition,

\begin{equation}
    \mathcal{H}_t(a_t) = \mathcal{H}_0(a_0)
\end{equation}



Then $s=0$ correspond to the point-like or bare particles, and recovering the original

combining with the equation \eqref{eq:effective_particle_operator}, and considering 
the parameter $t$ instead of $s$, the condition becomes,

\begin{equation}
    \mathcal{H}_t(a_0) = \mathcal{U}_t^\dagger\mathcal{H}_0(a_0) \mathcal{U}_t,
\end{equation}
differentiating with respect of $t$, 

\begin{equation}
    \mathcal{H}_t^\prime (a_0) \equiv \frac{d}{dt}\mathcal{H}_t(a_0) = 
    \sbrackets{-\mathcal{U}_t^\dagger \mathcal{U}_t^\prime, \mathcal{H}_t(a_0)} 
    = \sbrackets{\mathcal{G}_t(a_0), \mathcal{H}_t(a_0)},
\end{equation}
where $\mathcal{G}_t$ is the RGPEP generator, this generator is defined as 

We will consider the generator from Ref. \cite{PEP}

\begin{equation}
    \mathcal{G}_t = \sbrackets{\mathcal{H}_f, \mathcal{H}_{Pt}},
\end{equation}
where $\mathcal{H}_f$, the free part of $\mathcal{H}_t$, and $\mathcal{H}_{Pt}$ is 
defined as function of the interacting term.

\begin{equation}
    =  \sbrackets{\sbrackets{\mathcal{H}_f, \mathcal{H}_{Pt}}, \mathcal{H}_t}
\end{equation}

where $\mathcal{H}_t$ is the Hamiltonian interested in solving.

The free Hamiltonian $\mathcal{H}_f$ is the part of $\mathcal{H}_0(a_0)$ that does 
not depend on the coupling constants, 


\subsection{Diagram representation}

From the expression of the Hamiltonian, different terms can be separated and correlate
to a diagram representation of the process, similar to the Feynman diagrams.




\section{Cases} \label{sec:cases}
\subsection{Scalar case}


\section{Code implementation} \label{sec:code}

\section{Diagrams obtained} \label{sec:diagrams}

\subsection{Scalar}

\subsection{QED}

\subsection{QCD}

\section{Conclusions and future work} \label{sec:conclusions}



% Referencias %%%%%%%%%%%%%%%%%%%%%%%%%%%%%%%%%%%%%%%%%%%%%%%%%%%%%%%%%%%%%%%%%
%\newpage

\bibliography{references} % Entries are in the references.bib file
\nocite{QFT}
\nocite{QCDG}

\end{document}

