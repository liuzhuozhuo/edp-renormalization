\documentclass[11pt,a4paper,twoside,pdf]{article}

% Paquetes (añade otros si los necesitas):

\usepackage[T1]{fontenc}
\usepackage[utf8]{inputenc}


\usepackage[pdftex,final]{graphicx}
\bibliographystyle{plain} % We choose the "plain" reference style

% Style package
% Font Package (Palatino)
\usepackage{mathpazo}
% Packages for specific capabilities
\usepackage{rotating} % for text rotation in tables
\usepackage{multirow} % for multirow in tables
\usepackage{subfigure} % Place subfigures in figure environment
% Packages for specific symbols
\usepackage{amssymb}
\usepackage{amsmath}
\usepackage{amsfonts}
\usepackage{eurosym} % Euro symbol
\usepackage{bbding} % for \XSolidBrush
\usepackage{pifont} % for \ding{55} (a check mark)
\usepackage{macro}


\usepackage{latexsym}
\usepackage{soul}
\usepackage{array}
%\usepackage{marvosym}
\usepackage{epsfig}
\usepackage{graphics}
\usepackage{amsfonts}
\usepackage{xspace}
\usepackage{color}
\usepackage{booktabs}
\usepackage{xtab}
\usepackage[colorlinks=true,urlcolor=blue,linkcolor=blue,citecolor=blue]{hyperref}
\numberwithin{equation}{section}


\linespread{1.05}

% TFG en inglés:
\usepackage[english]{babel} 
\addto\captionsenglish{\renewcommand{\chaptername}{}}

% TFG en español:
%\usepackage[spanish,es-nodecimaldot,es-tabla,es-lcroman,es-nosectiondot,es-noindentfirst]{babel}
%\renewcommand\spanishchaptername{}

% Formato de la página:
\usepackage{fancyhdr}
\usepackage[top=2.88cm,bottom=2.97cm,left=2.95cm,right=2.95cm]{geometry}
\setlength{\parskip}{0.1cm}

% Pon aquí tus definiciones:

\newcommand{\dis}{\displaystyle}
%\sodef\an{}{.2em}{1em plus1em}{2em plus.1em minus.1em}

\begin{document}

% Portada %%%%%%%%%%%%%%%%%%%%%%%%%%%%%%%%%%%%%%%%%%%%%%%%%%%%%%%%%%%%%%%%%%%%%%

\pagestyle{empty}


\noindent
\begin{tabular}{r}
\includegraphics[width=8.8cm]{escudoUGRmonocromo.png} \\[-1.8ex]
\hspace{31mm}\vspace{-8mm}
\begin{tabular}{c}
\hline\\[-1ex]\hskip-2mm
{\bf Facultad de Ciencias}\hspace{18mm}
\end{tabular}
\end{tabular}

\large
\vspace{30mm}
\hspace{25mm}
\begin{tabular}{l}
GRADO EN F\'ISICA
\end{tabular}

\vspace{45mm}
\hspace{25mm}
\begin{tabular}{l}
TRABAJO FIN DE GRADO
\\[1.5ex]
\LARGE\bf Herramientas para cálculos \\ 
\LARGE\bf perturbativos en renormalización \\
\LARGE\bf de Hamiltonianos \\
\end{tabular}
%
\vfill
\hspace{25mm}
\begin{tabular}{l}
Presentado por:
\\
{\bf D. ZhuoZhuo Liu}
\\[3ex]
Curso Académico 2024/2025
\end{tabular}
%

\newpage
%
\begin{center}
{\bf Resumen}
\bigskip

\begin{minipage}{0.8\linewidth}
bla bla bla bla bla bla bla bla bla bla bla bla bla bla bla
bla bla bla bla bla bla bla bla bla bla bla bla bla bla bla
bla bla bla bla bla bla bla bla bla bla bla bla bla bla bla 
bla bla bla bla bla bla bla bla bla bla bla bla bla bla bla
bla bla bla bla bla bla bla bla bla bla bla bla bla bla bla
\end{minipage}

\newpage

{\bf Abstract} 
\bigskip

\begin{minipage}{0.8\linewidth}
bla bla bla bla bla bla bla bla bla bla bla bla bla bla bla
bla bla bla bla bla bla bla bla bla bla bla bla bla bla bla
bla bla bla bla bla bla bla bla bla bla bla bla bla bla bla 
bla bla bla bla bla bla bla bla bla bla bla bla bla bla bla
bla bla bla bla bla bla bla bla bla bla bla bla bla bla bla
\end{minipage}

\newpage

\end{center}

% Indice %%%%%%%%%%%%%%%%%%%%%%%%%%%%%%%%%%%%%%%%%%%%%%%%%%%%%%%%%%%%%%%%%%%%%%%
%\newpage

\tableofcontents

% Texto %%%%%%%%%%%%%%%%%%%%%%%%%%%%%%%%%%%%%%%%%%%%%%%%%%%%%%%%%%%%%%%%%%%%%%%%
%\newpage

\pagestyle{fancy}
\fancyhead[RO,LE]{\leftmark}
\fancyhead[LO,RE]{\thepage}
\fancyfoot{}

\newpage

\section{Introduction}


It's in our interest to formulate a theory compatible with special relativity and
quantum mechanics, this culminates in quantum field theories. 

we should expect the most general theories to be scale invariant, 

In the context of theoretical physics renormalization is an important procedure 

Tipical renormalization procedures apply to Lagrangian dynamics, 

\newpage

\section{Theoretical background}

To describe the relativistic interactions of effective particles in the framework of 
the renormalization group procedure for effective particles, making use of quantum 
theory in the front form of Hamiltonian dynamics. 

\subsection{Canonical Hamiltonian}

This is the Hamiltonian that describes the theory, it dictates the evolution and 
interactions of the elements in the theory. In 

In the case of field theories, a Hamiltonian density can be defined such that 
integrating over space coordinates the Hamiltonian is obtained. 

\begin{equation}
    H = \int d^3x \mathcal{H}
\end{equation}
for simplicity we will call the Hamiltonian density $\mathcal{H}$, as Hamiltonian.

Solving the system means diagonalizing the Hamiltonian matrix, and finding the 
eigenvalues, or in this context eigenenergies of the system. In general this is a 
non trivial process.

In field theory calculations the fundamental quantities to consider are the 
energy-momentums, angular momentums, and boosts, due to it's relation or conservation
under the Poincaré group, this is a set of transformation.

It's expression in point form of dynamics, this is expressing the set of dynamical 
variables refering to the physical condition of the system at some instant of time, 
we could say this is the "usual way" of expressing the dynamical variables, are,




\subsection{Fock space}

In order to describe a varible number of particles, the theory needs to be build 
in the Fock space, this is of the form,


\subsection{Front form of Hamiltonian Dynamics}

The front form of dynamics introduced by Dirac (1949) \cite{dirac_front_forms_1949} 
offers a couple advantages to the 

The quantization hypersurface considered is,

\begin{equation}
    x^+ = t+z = x^0 + x^3 = 0,
\end{equation}
then the rest of coordinates will be defined as

\begin{equation}
    x^- = x^0-x^3, \quad x^\perp = \{x^1, x^2\},
\end{equation}

In this set of coordinates, the fundamental quantities 

\subsection{RGPEP}

The solution 

RGPEP introduces effective particle operators (namely creation and annihilation 
operators) that differs from the canonical ones by a unitary transformation

\begin{equation}
    a_s = \mathcal{U}_sa_0\mathcal{U}_s^\dagger,
    \label{eq:effective_particle_operator}
\end{equation}
labeled by the parameter $s$, the renormalization group parameter. This parameter 
has dimension of length. Physically $s$ has the interpretation of the characteristic 
size of the effective particles.

Then $s=0$ correspond to the point-like or bare particles

Due to dimensional and notational reasons, it's convenient to use the scale parameter 
$t = s^4$ instead of $s$.

The effective Hamiltonian is related to the regulated canonical one with counter-terms 
by the condition,

\begin{equation}
    \mathcal{H}_t(a_t) = \mathcal{H}_0(a_0)
\end{equation}
combining with the equation \eqref{eq:effective_particle_operator}, and considering 
the parameter $t$ instead of $s$, the condition becomes,

\begin{equation}
    \mathcal{H}_t(a_0) = \mathcal{U}_t^\dagger\mathcal{H}_0(a_0) \mathcal{U}_t,
\end{equation}
differentiating with respect of $t$, 

\begin{equation}
    \mathcal{H}_t^\prime (a_0) \equiv \frac{d}{dt}\mathcal{H}_t(a_0) = 
    \sbrackets{-\mathcal{U}_t^\dagger \mathcal{U}_t^\prime, \mathcal{H}_t(a_0)} 
    = \sbrackets{\mathcal{G}_t(a_0), \mathcal{H}_t(a_0)},
\end{equation}
where $\mathcal{G}_t$ is the RGPEP generator, this generator is defined as 

We will consider the generator from Ref. \cite{PEP}

\begin{equation}
    \mathcal{G}_t = \sbrackets{\mathcal{H}_f, \mathcal{H}_{Pt}},
\end{equation}
where $\mathcal{H}_f$, the free part of $\mathcal{H}_t$, and $\mathcal{H}_{Pt}$ is 
defined as function of the interacting term.

\begin{equation}
    =  \sbrackets{\sbrackets{\mathcal{H}_f, \mathcal{H}_{Pt}}, \mathcal{H}_t}
\end{equation}

where $\mathcal{H}_t$ is the Hamiltonian interested in solving, 

\subsection{Counterterms}

The counterterms is an additional term added to the initial or bare Hamiltonian 
$\mathcal{H}_0$ to deal with the divergences due to loops, produced during the 
process. This way, ensure that the effective Hamiltonian $\mathcal{H}_t$ remains 
finite at all values of $t$.

\subsection{Diagram representation}

From the expression of the Hamiltonian, different terms can be separated and correlate
to a diagram representation of the process, similar to the Feymann diagrams.


\section{Cases}

\subsection{Scalar case}


\section{Code implementation}

\section{Diagrams obtained}

\subsection{Scalar}

\subsection{QED}

\subsection{QCD}

\section{Conclusions and future work}



% Referencias %%%%%%%%%%%%%%%%%%%%%%%%%%%%%%%%%%%%%%%%%%%%%%%%%%%%%%%%%%%%%%%%%
%\newpage

\bibliography{references} % Entries are in the references.bib file
\nocite{QFT}
\nocite{QCDG}

\end{document}

